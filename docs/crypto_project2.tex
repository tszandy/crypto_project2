\documentclass[titlepage]{article}
\usepackage[utf8]{inputenc}
\usepackage{listings}
\usepackage{amsmath}
\usepackage{enumitem}
\newcommand{\Mod}[1]{\ (\text{mod}\ #1)}

\title{Math 571-01, Cryptography Project 02 \\ Quadratic Sieve \\ University of Massachusetts, Amherst}
\author{Matthew Gramigna \\ Wei Xie \\ Barry Greengus}
\date{\today}


\begin{document}
	
	\maketitle
	
	\section{Introduction}
	The quadratic sieve is an efficient means of finding many numbers greater than the square root of a given $N$ whos squares modulus $N$ are $B-smooth$ for a given positive integer $B$. i.e\\
	\\
	Let $N,B\in{Z}\\
	$ find $a^2 \mod(N)$ s.t. $\forall p_i $ in $a^2 = p^{e_1}_1*p^{e_2}_2*...*p^{e_k}_k$, $p_i \leq B$\\
	\\
	\subsection{Difference by Squares Factoring}
	Why is this useful? Consider the problem of factoring, specifically the method of factoring using difference of squares. If a number $N$ is know to be the difference of two squares, say $X = Y^2 - Z^2$, then $X = (Y+Z) (Y-Z)$. So all we have to do to factor $N$ is to find a number $b$ such that $N + b^2$ is a perfect square. Then $N + b^2 = a^2$, so \[N = a^2 - b^2 = (a+b)(a-b)\] and we have just factored N.\\
	\\
	A random value of $b$ is unlikely to produce a perfect square, but it is fairly likely for a multiple $k$ of $N$ to equal the difference of two squares \[kN = a^2 - b^2 = (a+b)(a-b)\] such that $(a+b)$ or $(a-b)$, besides for being a factor of $kN$, is also a non-trivial factor of $N$. This means we only need to find a difference of two squares that equals a mutiple of $N$, which is the equivalent of finding $a$ and $b$ such that $a^2\equiv b^2 \pmod{N}$.
	\\
	This fact enables a three step factoring algorithm comprised of:\\
	\textbf{Step1}:\\
	\textbf{Step2}: \\
	\textbf{Step3}: \\
	
	Quadratic sieve solves the first step of this algorithm.

	\section{Overview of Quadratic Sieve}
	Explain how quadratic sieve works TODO more info +equations + fix "=" to congruent

		\noindent\textbf{Setup Step}: Given number N and set of primes P, where all elemenet in P >= B, set a=floor of the sqrt(N), set a quadratic polynomial. we will use F(T) = T\^2 - 221.\\

		\noindent\textbf{Step 1}: build a list of F(a) to F(L(a)). TODO define L(), explain why we use it. explain why we start at a.\\

		\noindent\textbf{Step 2}: For i=2 to B, where i=some p in P or is prime factor of some p in P:\\
	
		\noindent\textbf{Step 3}: Predict where division of elem in list by i CAN happen.\\
		if p | F(T), then T\^2 = N mod p has a solution, else no solution so you cant divide by p.\\
		So, if p odd  and T\^2 = N mod p has two solutions, a and b. all mulitples of those solutions can also be divided by the p\\\

		\noindent\textbf{Step 4}: Divide all multiples of the solutions a and b in the list by p\\

		\noindent\textbf{Step 5}: Whenever the quotient of a list element is 1,  it's prime factors are clearly only primes <= B and is thus B-smooth\\ 
	  
	\section{Implementation}
	We used GP, TODO more info
	
		\subsection{Initial Approach}
	
		\subsection{Final Implementation}
	
		\subsection{Interesting Details/etc?}
	
		\subsection{Testing}
	
	\section{Efficiency}
	
	\section{Source Code}
	
	\section{Group Organization/Administrative}
	
		\subsection{Git}
		TODO, we used git bc useful for XYZ

		\subsection{Meetings}
		met how often? helpful bc why?
	
\end{document}
